\documentclass[11pt]{article}
\usepackage[a4paper, margin=1in]{geometry}
%\usepackage{cite}
\usepackage[square]{natbib} % <- new

\usepackage{datetime}
%\longdate

\usepackage{titlesec}
\titleformat*{\section}{\large\bfseries}

\usepackage{parskip}
\setlength{\parindent}{0pt}

\begin{document}

\title{Annotated bibliography - Nichupt{\'e} lagoon processes}
\author{M.W.}
\date{\today}
\maketitle

\section*{Background on Laguna Nichupt{\'e}}

\citet{merino90} made measurements in 1982 and 1983 in the lagoon. \citet{pedrozoacuna08} did a modeling study for a Masters project. 

Other work in Nichupte or Bojorquez: oxygen and eutrophication in Bojorquez \citep{reyes91}, on tourism in the area \citep{torres05}, more eutrophication in response to tourism \citep{merino92}, zooplankton in Bojorquez \citep{alvarezcadena96}, sediment oxygen demand in Bojorquez \citep{valdeslozano06}.

Other Yucatan coastal lagoons (but not including Nichupte) \citep{herrerasilveria98}, 


\section*{Importance of the coral reefs in the area}

\citet{villegassanchez14} showed that the Puerto Morelos coral reef shows high connectivity with one in the Gulf of Mexico, and work has also been done showing connectivity with a reef in Puerto Rico \citep{labastidaestrada14}.





\bibliography{nichupte}{}
\bibliographystyle{apalike}
\end{document}
