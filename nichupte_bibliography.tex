\documentclass[11pt]{article}
\usepackage[a4paper, margin=1in]{geometry}
%\usepackage{cite}
\usepackage[square]{natbib} % <- new

\usepackage{datetime}
%\longdate

\usepackage{titlesec}
\titleformat*{\section}{\large\bfseries}

\usepackage{parskip}
\setlength{\parindent}{0pt}

\begin{document}

\title{Annotated bibliography - Nichupt{\'e} lagoon processes}
\author{M.W.}
\date{\today}
\maketitle

\subsection*{Background on Laguna Nichupt{\'e}}

\citet{merino90} made measurements in 1982 and 1983 in the lagoon. \citet{pedrozoacuna08} did a modeling study for a Masters project. Oxygen and eutrophication in Bojorquez \citep{reyes91}, on tourism in the area \citep{torres05}, more eutrophication in response to tourism \citep{merino92}, zooplankton in Bojorquez \citep{alvarezcadena96}, sediment oxygen demand in Bojorquez \citep{valdeslozano06}.

\subsubsection{2018 ECSS paper}
\citet{romerosierra18} (ECSS): 

This paper has an extensive literature review of Nichupté Lagoon System (NLS) papers, including several by M. Merino et al. that are in conference proceedings. 

Water sampling in the NLS: surface (<0.5m) and bottom water samples made at 13 stations, profiles with a Seabird CTD at 28 locations. 

Numerical modeling of the NLS was with ROMS. They fix the coastal depth to 50cm (a limitation of the model or the bathymetry, I'm not sure), fix the T and S to be 20C and 35 PSU, and have closed inlets. Then the model was forced with North American Regional Reanalysis (NARR) wind stresses. Model initialized with the "last output from a 2 year-long climatological simulation." Simulations were January 2015 - June 2016, analyzed August 2015 for the rainy season and March 2016 for the cold-front season. "Thus the model's circulation results for these two periods complement the observational results..." so it seems that no observations actually went into the model. They do some passive tracer experiments in the model- to determine a deviation from the initial state (sigma, euqation 1) and determine residence times (tau, equation 2), though I struggle with what this means if the lagoon is closed to the ocean. 

Results: Temperature was 30-34 C (rainy) and 26-29 C (cold-front) from measuremnets. This is figure two - are the top and bottom measurements shown (maybe?) - if so there is thermal stratification during the rainy measurements, none during cold-front measurements. Stations 6 and 28 seem to be missing from the map (figure 1b), so I'm not sure where those measurements were. 28 is only measured on the Rainy sampling. But, site 27 is just south of the south inlet, site 26 is inside the south inlet, site 2 is just south of the north inlet (inside). LN1 is near site 2, and site SH is in the south inlet. (These are water smample staitons, the temp/salinity stations are 1-28).

Measurements in general show higher salinity in the south than in the north. If there is freshwater flowing into the lagoon, this is in agreement with the general flow direction being northward if water of ocean salinity enters the southern inlet. The lagoon is MUCH fresher during the cold-front season than the rainy season. (Why???) 

Dissolved oxygen concentrations measured were 2.6 - 5.6 mg/l in the rainy season and 2.8-6.8 mg/l in the cold-front season. Really low DO water coming from Rio Inglés (that's how Merino 1990 calls the basin - it isn't shown in this paper). A pH of 8.0-8.8 was measured. 

(From Merino 1990: Rio Inglés has S>37 in May and the 8-11 in June: evaporation and wetland runoff into this basin are strong).

Hydrodynamics: "flow mostly northwestward" because of the easterly trade winds. Suggests we also need to include wind in our idealized model. The "wind stress is roughly four times stronger in the cold-front season" which accounds for stronger currents then than in the rainy season, though the patterns are the same.

There is submarine discharge coming into the NLS. How do we account for that in our work? Is it too big of an unknown? How much could enter and still maintain our story? Does it matter...?

Limitations of this work: the numerical model does not appear to be validated, and does not appear to have open inlets. I don't entirely understand how they calculate residence time in the NLS without flow through the inlets. Positives of this work: measurements of nutrients and DO, and an extensive literature review.  

\subsubsection*{Other Yucatan lagoons:}
Other Yucatan coastal lagoons (but not including Nichupte) \citep{herrerasilveria98}, 


\subsection*{Importance of the coral reefs in the area}

\citet{villegassanchez14} showed that the Puerto Morelos coral reef shows high connectivity with one in the Gulf of Mexico, and work has also been done showing connectivity with a reef in Puerto Rico \citep{labastidaestrada14}.

\subsection*{Non-tidal controls on water level in lagoons and estuaries}

\citet{ocallaghan07} saw variability in the estuarine water level by remotely forced continental shelf waves in an Australian estuary.


\bibliography{nichupte}{}
\bibliographystyle{apalike}
\end{document}
